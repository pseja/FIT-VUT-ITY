\documentclass[11pt, twocolumn]{article}[15.03.2024]

\usepackage[a4paper, left=1.4cm, top=2.3cm, text={18.3cm, 25.2cm}]{geometry}
\usepackage[czech]{babel}
\hyphenation{Mea-ly-ho}
%\usepackage[hidelinks]{hyperref}
\usepackage{amsthm}
\usepackage{amsmath}
\usepackage{lmodern}
\usepackage[utf8]{inputenc}
\usepackage[T1]{fontenc}

\begin{document}
\begin{titlepage}
    \begin{center}
        \Huge{\textsc{Vysoké učení technické v~Brně\\[0.5em]\huge{Fakulta informačních technologií}}}
        \vspace{\stretch{0.382}}
        \LARGE{\\Typografie a~publikování\,--\,2. projekt\\[0.6em]Sazba dokumentů a~matematických výrazů}
        \vspace{\stretch{0.618}}
    \end{center}
    {\Large 2024 \hfill Lukáš Pšeja (xpsejal00)}
\end{titlepage}

\section*{Úvod} \label{strana1}
    V~této úloze si vysázíme titulní stranu a~kousek matematického textu, v~němž se vyskytují například Definice~\ref{definice1} nebo rovnice~\eqref{rovnice2} na straně~\pageref{strana1}. Pro vytvoření těchto odkazů používáme kombinace příkazů \verb|\label|, \verb|\ref|, \verb|\eqref| a~\verb|\pageref|. Před odkazy patří nezlomitelná mezera. Pro zvýrazňování textu se používají příkazy \verb|\verb| a~\verb|\emph|.

    Titulní strana je vysázena prostředím \verb|titlepage| a~nadpis je v~optickém středu s~využitím \emph{přesného} zlatého řezu, který byl probrán na přednášce. Dále jsou na titulní straně čtyři různé velikosti písma a~mezi dvojicemi řádků textu je použito řádkování se zadanou relativní velikostí 0{,}5\,em a~0{,}6\,em\footnote{Použijte správný typ mezery mezi číslem a~jednotkou.}.

\section{Matematický text}
    Matematické symboly a~výrazy v~plynulém textu jsou v~prostředí \verb|math|. Definice a~věty sázíme v~prostředí definovaném příkazem \verb|\newtheorem| z~balíku \verb|amsthm|. Tato prostředí obracejí význam \verb|\emph|: uvnitř textu sázeného kurzívou se zvýrazňuje písmem v~základním řezu. Někdy je vhodné použít konstrukci \verb|${}$| nebo \verb|\mbox{}|, která zabrání zalomení (matematického) textu. Pozor také na tvar i~sklon řeckých písmen: srovnejte \verb|\epsilon| a~\verb|\varepsilon|, \verb|\Xi| a~\verb|\varXi|.

    \newtheorem{definition}{Definice}\label{definice1}
    \begin{definition}
        \emph{Konečný přepisovací stroj} neboli \emph{Mealyho automat} je definován jako uspořádaná pětice tvaru $M=(Q, \varSigma, \varGamma, \delta, q_0)$, kde:
        \begin{itemize}
            \item[$\bullet$] $Q$ je konečná množina \emph{stavů},
            \item[$\bullet$] $\varSigma$ je konečná \emph{vstupní abeceda},
            \item[$\bullet$] $\varGamma$ je konečná \emph{výstupní abeceda},
            \item[$\bullet$] $\delta : Q \times \varSigma \to Q \times \varGamma$ je totální \emph{přechodová funkce},
            \item[$\bullet$] $q_0 \in Q$ je \emph{počáteční stav}.
        \end{itemize}
    \end{definition}

\subsection{Podsekce s~definicí}
    Pomocí přechodové funkce $\delta$ zavedeme novou funkci~$\delta^*$ pro překlad vstupních slov $u \in \varSigma^*$ do výstupních slov $w \in \varGamma^*$.

    \begin{definition} \label{definice2}
        Nechť $M=(Q, \varSigma, \varGamma, \delta, q_0)$ je Mealyho automat. \emph{Překládací funkce} $\delta^* : Q \times \varSigma^* \times \varGamma^* \to \varGamma^*$ je pro každý stav $q \in Q$, symbol $x \in \varSigma$, slova $u \in \varSigma^*$, $w \in \varGamma^*$ definována rekurentním předpisem:
        \begin{itemize}
            \item[$\bullet$] $\delta^*(q, \varepsilon, w) = w$
            \item[$\bullet$] $\delta^*(q, xu, w) = \delta^*(q^\prime, u, wy)$, kde $(q^\prime, y) = \delta(q, x)$
        \end{itemize}
    \end{definition}

\subsection{Rovnice}
    Složitější matematické formule sázíme mimo plynulý text pomocí prostředí \verb|displaymath|. Lze umístit i~více výrazů na jeden řádek, ale pak je třeba tyto vhodně oddělit, například pomocí \verb|\quad|, při dostatku místa i~\verb|\qquad|.
    \begin{displaymath} \label{trifunkce}
        g^{a_n} \notin A^{B^n}\qquad
        y^{1}_{0} - \sqrt[5]{x + \sqrt[7]{y}}\qquad
        x > y^2 \geq y^3
    \end{displaymath}

    Velikost závorek a~svislých čar je potřeba přizpůsobit jejich obsahu. Velikost lze stanovit explicitně, anebo pomocí \verb|\left| a~\verb|\right|. Kombinační čísla sázejte makrem \verb|\binom|.
    $$\left|\bigcup P\right|=\sum\limits_{\emptyset \neq X \subseteq P} (-1)^{|X|-1}\left|\bigcap X\right|$$

    $$F_{n+1}=\binom{n}{0}+\binom{n-1}{1}+\binom{n-2}{2}+\cdots+\binom{\lceil\frac{n}{2}\rceil}{\lfloor\frac{n}{2}\rfloor}$$

    V~rovnici~\eqref{rovnice1} jsou tři typy závorek s~různou \emph{explicitně} definovanou velikostí. Obě rovnice mají svisle zarovnaná rovnítka. Použijte k~tomu vhodné prostředí.
    \begin{eqnarray}
        \label{rovnice1}
        \biggl(\Bigl\{b \otimes \bigl[c_1 \oplus c_2\bigr] \circ a\Bigr\}^\frac{2}{3}\biggr) & = & \log_z x\\
        \label{rovnice2}
        \int^b_a f(x)\,dx & = & -\int^a_b f(y)\,dy
    \end{eqnarray}
    V~této větě vidíme, jak se vysází proměnná určující limitu v~běžném textu: $\lim_{m \to \infty} f(m)$. Podobně je to i~s~dalšími symboly jako $\bigcup_{N \in \mathcal{M}} N$ či $\sum^{m}_{i=1} x^{2}_{i}$. S~vynucením méně úsporné sazby příkazem \verb|\limits| budou vzorce vysázeny v~podobě $\lim\limits_{m\to\infty} f(m)$ a~$\sum\limits^{m}_{i=1} x^{2}_{i}$.

\section{Matice}
    Pro sázení matic se používá prostředí \verb|array| a~závorky s~výškou nastavenou pomocí \verb|\left|, \verb|\right|.
    $$
        D = \left|\begin{array}{cccc}
             a_{11} & a_{12} & \cdots & a_{1n} \\
             a_{21} & a_{22} & \cdots & a_{2n} \\
             \vdots & \vdots & \ddots & \vdots \\
             a_{m1} & a_{m2} & \cdots & a_{mn}
        \end{array}\right|=\left|\begin{array}{cc}
             x & y \\
             t & w
        \end{array}\right|=xw - yt
    $$
    
    Prostředí \verb|array| lze úspěšně využít i~jinde, například na pravé straně následující rovnosti.
    $$
        \binom{n}{k}=\left\{\begin{array}{ll}
             \frac{n!}{k!(n-k)!} & \mathrm{pro}\ 0 \leq k~\leq n \\
              0 & \mathrm{jinak}
        \end{array}\right.
    $$
\end{document}